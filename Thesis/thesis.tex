\documentclass[8pt]{extarticle}

%lhs2TeX imports -- don't remove!
%% ODER: format ==         = "\mathrel{==}"
%% ODER: format /=         = "\neq "
%
%
\makeatletter
\@ifundefined{lhs2tex.lhs2tex.sty.read}%
  {\@namedef{lhs2tex.lhs2tex.sty.read}{}%
   \newcommand\SkipToFmtEnd{}%
   \newcommand\EndFmtInput{}%
   \long\def\SkipToFmtEnd#1\EndFmtInput{}%
  }\SkipToFmtEnd

\newcommand\ReadOnlyOnce[1]{\@ifundefined{#1}{\@namedef{#1}{}}\SkipToFmtEnd}
\usepackage{amstext}
\usepackage{amssymb}
\usepackage{stmaryrd}
\DeclareFontFamily{OT1}{cmtex}{}
\DeclareFontShape{OT1}{cmtex}{m}{n}
  {<5><6><7><8>cmtex8
   <9>cmtex9
   <10><10.95><12><14.4><17.28><20.74><24.88>cmtex10}{}
\DeclareFontShape{OT1}{cmtex}{m}{it}
  {<-> ssub * cmtt/m/it}{}
\newcommand{\texfamily}{\fontfamily{cmtex}\selectfont}
\DeclareFontShape{OT1}{cmtt}{bx}{n}
  {<5><6><7><8>cmtt8
   <9>cmbtt9
   <10><10.95><12><14.4><17.28><20.74><24.88>cmbtt10}{}
\DeclareFontShape{OT1}{cmtex}{bx}{n}
  {<-> ssub * cmtt/bx/n}{}
\newcommand{\tex}[1]{\text{\texfamily#1}}	% NEU

\newcommand{\Sp}{\hskip.33334em\relax}


\newcommand{\Conid}[1]{\mathit{#1}}
\newcommand{\Varid}[1]{\mathit{#1}}
\newcommand{\anonymous}{\kern0.06em \vbox{\hrule\@width.5em}}
\newcommand{\plus}{\mathbin{+\!\!\!+}}
\newcommand{\bind}{\mathbin{>\!\!\!>\mkern-6.7mu=}}
\newcommand{\rbind}{\mathbin{=\mkern-6.7mu<\!\!\!<}}% suggested by Neil Mitchell
\newcommand{\sequ}{\mathbin{>\!\!\!>}}
\renewcommand{\leq}{\leqslant}
\renewcommand{\geq}{\geqslant}
\usepackage{polytable}

%mathindent has to be defined
\@ifundefined{mathindent}%
  {\newdimen\mathindent\mathindent\leftmargini}%
  {}%

\def\resethooks{%
  \global\let\SaveRestoreHook\empty
  \global\let\ColumnHook\empty}
\newcommand*{\savecolumns}[1][default]%
  {\g@addto@macro\SaveRestoreHook{\savecolumns[#1]}}
\newcommand*{\restorecolumns}[1][default]%
  {\g@addto@macro\SaveRestoreHook{\restorecolumns[#1]}}
\newcommand*{\aligncolumn}[2]%
  {\g@addto@macro\ColumnHook{\column{#1}{#2}}}

\resethooks

\newcommand{\onelinecommentchars}{\quad-{}- }
\newcommand{\commentbeginchars}{\enskip\{-}
\newcommand{\commentendchars}{-\}\enskip}

\newcommand{\visiblecomments}{%
  \let\onelinecomment=\onelinecommentchars
  \let\commentbegin=\commentbeginchars
  \let\commentend=\commentendchars}

\newcommand{\invisiblecomments}{%
  \let\onelinecomment=\empty
  \let\commentbegin=\empty
  \let\commentend=\empty}

\visiblecomments

\newlength{\blanklineskip}
\setlength{\blanklineskip}{0.66084ex}

\newcommand{\hsindent}[1]{\quad}% default is fixed indentation
\let\hspre\empty
\let\hspost\empty
\newcommand{\NB}{\textbf{NB}}
\newcommand{\Todo}[1]{$\langle$\textbf{To do:}~#1$\rangle$}

\EndFmtInput
\makeatother
%
%
%
%
%
%
% This package provides two environments suitable to take the place
% of hscode, called "plainhscode" and "arrayhscode". 
%
% The plain environment surrounds each code block by vertical space,
% and it uses \abovedisplayskip and \belowdisplayskip to get spacing
% similar to formulas. Note that if these dimensions are changed,
% the spacing around displayed math formulas changes as well.
% All code is indented using \leftskip.
%
% Changed 19.08.2004 to reflect changes in colorcode. Should work with
% CodeGroup.sty.
%
\ReadOnlyOnce{polycode.fmt}%
\makeatletter

\newcommand{\hsnewpar}[1]%
  {{\parskip=0pt\parindent=0pt\par\vskip #1\noindent}}

% can be used, for instance, to redefine the code size, by setting the
% command to \small or something alike
\newcommand{\hscodestyle}{}

% The command \sethscode can be used to switch the code formatting
% behaviour by mapping the hscode environment in the subst directive
% to a new LaTeX environment.

\newcommand{\sethscode}[1]%
  {\expandafter\let\expandafter\hscode\csname #1\endcsname
   \expandafter\let\expandafter\endhscode\csname end#1\endcsname}

% "compatibility" mode restores the non-polycode.fmt layout.

\newenvironment{compathscode}%
  {\par\noindent
   \advance\leftskip\mathindent
   \hscodestyle
   \let\\=\@normalcr
   \let\hspre\(\let\hspost\)%
   \pboxed}%
  {\endpboxed\)%
   \par\noindent
   \ignorespacesafterend}

\newcommand{\compaths}{\sethscode{compathscode}}

% "plain" mode is the proposed default.
% It should now work with \centering.
% This required some changes. The old version
% is still available for reference as oldplainhscode.

\newenvironment{plainhscode}%
  {\hsnewpar\abovedisplayskip
   \advance\leftskip\mathindent
   \hscodestyle
   \let\hspre\(\let\hspost\)%
   \pboxed}%
  {\endpboxed%
   \hsnewpar\belowdisplayskip
   \ignorespacesafterend}

\newenvironment{oldplainhscode}%
  {\hsnewpar\abovedisplayskip
   \advance\leftskip\mathindent
   \hscodestyle
   \let\\=\@normalcr
   \(\pboxed}%
  {\endpboxed\)%
   \hsnewpar\belowdisplayskip
   \ignorespacesafterend}

% Here, we make plainhscode the default environment.

\newcommand{\plainhs}{\sethscode{plainhscode}}
\newcommand{\oldplainhs}{\sethscode{oldplainhscode}}
\plainhs

% The arrayhscode is like plain, but makes use of polytable's
% parray environment which disallows page breaks in code blocks.

\newenvironment{arrayhscode}%
  {\hsnewpar\abovedisplayskip
   \advance\leftskip\mathindent
   \hscodestyle
   \let\\=\@normalcr
   \(\parray}%
  {\endparray\)%
   \hsnewpar\belowdisplayskip
   \ignorespacesafterend}

\newcommand{\arrayhs}{\sethscode{arrayhscode}}

% The mathhscode environment also makes use of polytable's parray 
% environment. It is supposed to be used only inside math mode 
% (I used it to typeset the type rules in my thesis).

\newenvironment{mathhscode}%
  {\parray}{\endparray}

\newcommand{\mathhs}{\sethscode{mathhscode}}

% texths is similar to mathhs, but works in text mode.

\newenvironment{texthscode}%
  {\(\parray}{\endparray\)}

\newcommand{\texths}{\sethscode{texthscode}}

% The framed environment places code in a framed box.

\def\codeframewidth{\arrayrulewidth}
\RequirePackage{calc}

\newenvironment{framedhscode}%
  {\parskip=\abovedisplayskip\par\noindent
   \hscodestyle
   \arrayrulewidth=\codeframewidth
   \tabular{@{}|p{\linewidth-2\arraycolsep-2\arrayrulewidth-2pt}|@{}}%
   \hline\framedhslinecorrect\\{-1.5ex}%
   \let\endoflinesave=\\
   \let\\=\@normalcr
   \(\pboxed}%
  {\endpboxed\)%
   \framedhslinecorrect\endoflinesave{.5ex}\hline
   \endtabular
   \parskip=\belowdisplayskip\par\noindent
   \ignorespacesafterend}

\newcommand{\framedhslinecorrect}[2]%
  {#1[#2]}

\newcommand{\framedhs}{\sethscode{framedhscode}}

% The inlinehscode environment is an experimental environment
% that can be used to typeset displayed code inline.

\newenvironment{inlinehscode}%
  {\(\def\column##1##2{}%
   \let\>\undefined\let\<\undefined\let\\\undefined
   \newcommand\>[1][]{}\newcommand\<[1][]{}\newcommand\\[1][]{}%
   \def\fromto##1##2##3{##3}%
   \def\nextline{}}{\) }%

\newcommand{\inlinehs}{\sethscode{inlinehscode}}

% The joincode environment is a separate environment that
% can be used to surround and thereby connect multiple code
% blocks.

\newenvironment{joincode}%
  {\let\orighscode=\hscode
   \let\origendhscode=\endhscode
   \def\endhscode{\def\hscode{\endgroup\def\@currenvir{hscode}\\}\begingroup}
   %\let\SaveRestoreHook=\empty
   %\let\ColumnHook=\empty
   %\let\resethooks=\empty
   \orighscode\def\hscode{\endgroup\def\@currenvir{hscode}}}%
  {\origendhscode
   \global\let\hscode=\orighscode
   \global\let\endhscode=\origendhscode}%

\makeatother
\EndFmtInput
%

\usepackage{extsizes}
\usepackage{fullpage}
\usepackage{amsmath,amsfonts,amssymb}
\usepackage{cite}
\usepackage{hyperref}
\usepackage{listings}
\usepackage{alltt}
\author{Ernesto Rodriguez}
\begin{document}
\Huge{\bf Generic Programming in F\#\\[1cm]}
\large{\bf Ernesto Rodriguez\\[0.5cm]}
\emph{Computer Science \\ Utrecht University \\ Utrecht \\ The Netherlands \\[0.5cm]}
\emph{Type: Master's Thesis Proposal \\ Date: November 28th, 2015 \\ Supervisor: Dr. Wouter Swierstra\\}
\line(1,0){520}\\ \\
\line(1,0){520}

\part{Background}
\section{Datatype Generic Programming}

Functional programming languages often use algebraic data types (ADTs)
in order to represent values. ADTs are defined by cases by providing a
type constructor for each case and the type of values that the
constructor accepts in order to produce a value of the ADT being
defined. In other words, a type constructor is a function that takes a
group of values of different types and produces a value of the type of
the ADT.

To define functions over ADTs, functional languages provide a
mechanism to deconstruct ADTs called pattern matching. This mechanism
allows the programmer to check wether a particular value was
constructed using the specified type constructor and allows the
programmer to extract the arguments used to produce the value. This
mechanism is very elegant since it allows to define functions by
induction, however it has several shortcommings.

A function that pattern matches a value over the constructors of a
particular ADT restricts that the type of the value being pattern
matched to be monomorphic. This leads to functions being implemented
multiple times -- either when a existing ADT is modified or a new ADT
is declared~\cite{polyp}. Since functional programming is all about
abstraction, several methods have been developed to allow functions to
become more polymorphic.

An ADT can be decalred to be higher-kinded. This means that a
definition of a list \ensuremath{\mathbf{data}\;\Conid{List}\mathrel{=}\Conid{Cons}\;\Conid{Int}\;\Conid{List}\;|\;\Conid{Nil}} can abstract
over its content and become \ensuremath{\mathbf{data}\;\Conid{List}\;\Varid{a}\mathrel{=}\Conid{Cons}\;\Varid{a}\;(\Conid{List}\;\Varid{a})\;|\;\Conid{Nil}}. Then a function, such as lenght, might de-construct the list
without performing any operations on its contents -- the type
represented by \ensuremath{\Varid{a}}. Such function can operate on a list of any type --
this is called parametric polymorphism. The programmer might also wish
to implement functions that operate on the contents of a list without
restricting the type of the list's content to be monomorphic. This can
be done by requiring that the function is also provided with a set of
operations that can be performed on its argument. For example, the
\ensuremath{\Varid{sum}} function could be implemented by requiring that a function to
add two elements of type \ensuremath{\Varid{a}} is provided. This is called ad-hoc
polymorphism.

Both of theese approaches can be used to define generic
functions. This is evidenced by the libraries Scrap your Boilerpate
Code~\cite{SYB} and Uniplate~\cite{Uniplate}. Both libraries specify a
family of operations that must be supported by a type and use ad-hoc
polymorphism to implement many generic functions (for example \ensuremath{\Varid{length}}
or \ensuremath{\Varid{increment}}) in terms of the family of operations. The programer
only needs to do pattern matching when defining theese base operations
and both libraries provide mechanisms to do it automatically.

Although both of theese features allow the definition of many generic
functions, a more general approach exists. Recall that every value
expressed as an ADT is a type constructor followed by values of other
types. For simplicity consider type constructors taking only one
argument, they look something like \ensuremath{\Conid{C}\;\Varid{a}}. One could then define
functions that generalize over \emph{every} type constructor. This is
the idea behind polytipic programming~\cite{polyp} which later evolved
into Datatype Generic Programming.

This method has matured over many years in Haskell alone there are
numerous tools and libraries for datatype generic programming,
including PolyP~\cite{polyp}, Generic Haskell~\cite{GenericHaskell},
RepLib~\cite{RepLib}, Regular~\cite{Regular},
Multi-Rec~\cite{multirec}, Instant Generics~\cite{instant2} and many
others. Some Haskell compilers even support Datatype Generic
Programming natively. The present thesis focuses on Datatype Generic
Programming and will use Regular~\cite{Regular} as a reference.

\subsection{Generic Programming with Regular}

The idea behind Datatype Generic Programming is to encode ADTs using
only a handful of datatypes, such encoding is called a
\emph{representation}. Each library uses different types to define
representations. Representations can encode many ADTs but not all of
them; the ADTs that are expressible by a representation is called the
\emph{universe}.

In the case of Regular, its universe are all the ADTs that:
\begin{itemize}
\item Have no type arguments (are of kind *)
\item Are not mutually recursive
\end{itemize}
This universe includes many common types like lists, trees and simple
DSLs but is smaller than the set of types definible in Haskell 98.

To represent its universe, regular uses the following types:
\begin{hscode}\SaveRestoreHook
\column{B}{@{}>{\hspre}l<{\hspost}@{}}%
\column{E}{@{}>{\hspre}l<{\hspost}@{}}%
\>[B]{}\mathbf{data}\;\Conid{K}\;\Varid{a}\;\Varid{r}\mathrel{=}\Conid{K}\;\Varid{a}{}\<[E]%
\\
\>[B]{}\mathbf{data}\;\Conid{Id}\;\Varid{r}\mathrel{=}\Conid{Id}\;\Varid{r}{}\<[E]%
\\
\>[B]{}\mathbf{data}\;\Conid{Unit}\;\Varid{r}\mathrel{=}\Conid{Unit}{}\<[E]%
\\
\>[B]{}\mathbf{data}\;(\Varid{f}\;\oplus\;\Varid{g})\;\Varid{r}\mathrel{=}\Conid{Inl}\;(\Varid{f}\;\Varid{r})\mid \Conid{Inr}\;(\Varid{g}\;\Varid{r}){}\<[E]%
\\
\>[B]{}\mathbf{data}\;(\Varid{f}\;\otimes\;\Varid{g})\;\Varid{r}\mathrel{=}\Varid{f}\;\Varid{r}\;\otimes\;\Varid{g}\;\Varid{r}{}\<[E]%
\ColumnHook
\end{hscode}\resethooks

The types have the following roles:
\begin{itemize}
\item \ensuremath{\Conid{K}} represents the occurence of values of primitive types (eg. \ensuremath{\Conid{Int}} or \ensuremath{\Conid{Bool}})
\item \ensuremath{\Conid{Id}} represents recursion over the same type
\item \ensuremath{\Conid{Unit}} represents a constructor which takes no arguments
\item \ensuremath{(\Varid{f}\;\oplus\;\Varid{g})} represents sum of two representations. This happens when a type has multiple constructors
\item \ensuremath{(\Varid{f}\;\otimes\;\Varid{g})} represents product of two representstions. This happens when a constructor takes multiple arguments.
\end{itemize}

To familiarize ourselves how a type can be encoded with a
representation, consider representing a list of \ensuremath{\Conid{Int}} with theese
types:
\begin{hscode}\SaveRestoreHook
\column{B}{@{}>{\hspre}l<{\hspost}@{}}%
\column{E}{@{}>{\hspre}l<{\hspost}@{}}%
\>[B]{}\mathbf{data}\;\Conid{List}\mathrel{=}\Conid{Cons}\;\Conid{Int}\;\Conid{List}\mid \Conid{Nil}{}\<[E]%
\ColumnHook
\end{hscode}\resethooks
This would be represented by the type:
\begin{hscode}\SaveRestoreHook
\column{B}{@{}>{\hspre}l<{\hspost}@{}}%
\column{E}{@{}>{\hspre}l<{\hspost}@{}}%
\>[B]{}\mathbf{type}\;\Conid{Rep}\mathrel{=}(\Conid{K}\;\Conid{Int}\;\otimes\;\Conid{Id})\;\oplus\;\Conid{Unit}{}\<[E]%
\ColumnHook
\end{hscode}\resethooks

It is straightforward to see that the sum of constructors gets
translated to the \ensuremath{\oplus} type. The \ensuremath{\otimes} type is used to join the
arguments of the first constructor. One of the arguments is a
primitive \ensuremath{\Conid{Int}} represented with \ensuremath{\Conid{K}\;\Conid{Int}} and the second arguments is a
recursive ocurrence of the list, represented by \ensuremath{\Conid{Id}}. Finally, the
\ensuremath{\Conid{Nil}} constructor is represented by \ensuremath{\Conid{Unit}}.

To write generic functions in Regular, one must declare a type as an
instance of the Regular typeclass. The Regular typeclass is the following:
\begin{hscode}\SaveRestoreHook
\column{B}{@{}>{\hspre}l<{\hspost}@{}}%
\column{3}{@{}>{\hspre}l<{\hspost}@{}}%
\column{E}{@{}>{\hspre}l<{\hspost}@{}}%
\>[B]{}\mathbf{class}\;(\Conid{Functor}\;(\Conid{PF}\;\Varid{a}))\Rightarrow \Conid{Regular}\;\Varid{a}\;\mathbf{where}{}\<[E]%
\\
\>[B]{}\hsindent{3}{}\<[3]%
\>[3]{}\mathbf{type}\;\Conid{PF}\;\Varid{a}\mathbin{::}\mathbin{*}\to \mathbin{*}{}\<[E]%
\\
\>[B]{}\hsindent{3}{}\<[3]%
\>[3]{}\Varid{from}\mathbin{::}\Varid{a}\to \Conid{PF}\;\Varid{a}\;\Varid{a}{}\<[E]%
\\
\>[B]{}\hsindent{3}{}\<[3]%
\>[3]{}\Varid{to}\mathbin{::}\Conid{PF}\;\Varid{a}\;\Varid{a}\to \Varid{a}{}\<[E]%
\ColumnHook
\end{hscode}\resethooks

The constituients of the class are a type called \ensuremath{\Conid{PF}} which simply is
the representation using the types of Regular and two functions \ensuremath{\Varid{to}}
and \ensuremath{\Varid{from}} that convert values to representations and representations
to values. For the case of the \ensuremath{\Conid{List}} provided above, an instance
could be the following:
\begin{hscode}\SaveRestoreHook
\column{B}{@{}>{\hspre}l<{\hspost}@{}}%
\column{3}{@{}>{\hspre}l<{\hspost}@{}}%
\column{E}{@{}>{\hspre}l<{\hspost}@{}}%
\>[B]{}\mathbf{instance}\;\Conid{Regular}\;\Conid{List}\;\mathbf{where}{}\<[E]%
\\
\>[B]{}\hsindent{3}{}\<[3]%
\>[3]{}\mathbf{type}\;\Conid{PF}\;\Conid{List}\mathrel{=}(\Conid{K}\;\Conid{Int}\;\otimes\;\Conid{Id})\;\oplus\;\Conid{Unit}{}\<[E]%
\\[\blanklineskip]%
\>[B]{}\hsindent{3}{}\<[3]%
\>[3]{}\Varid{from}\;(\Conid{Cons}\;\Varid{i}\;\Varid{l})\mathrel{=}\Conid{Inl}\;(\Conid{K}\;\Varid{i}\;\otimes\;\Conid{Id}\;\Varid{l}){}\<[E]%
\\
\>[B]{}\hsindent{3}{}\<[3]%
\>[3]{}\Varid{from}\;\Conid{Nil}\mathrel{=}\Conid{Inr}\;\Conid{Unit}{}\<[E]%
\\[\blanklineskip]%
\>[B]{}\hsindent{3}{}\<[3]%
\>[3]{}\Varid{to}\;(\Conid{Inl}\;(\Conid{K}\;\Varid{i}\;\otimes\;\Conid{Id}\;\Varid{l}))\mathrel{=}\Conid{Cons}\;\Varid{i}\;\Varid{l}{}\<[E]%
\\
\>[B]{}\hsindent{3}{}\<[3]%
\>[3]{}\Varid{to}\;(\Conid{Inr}\;\Conid{Unit})\mathrel{=}\Conid{Nil}{}\<[E]%
\ColumnHook
\end{hscode}\resethooks

This instance declaration is very unremarkable. In general, providing
an instance of Regular for a particular type is a mechanical process
and Template Haskell~\cite{TemplateHaskell} is used to automatically
derive them.

Generic functions can now be expressed in terms of values of
representation types instead of using values of the type itself. A
generic function is expressed as a typeclass and is implemented by
creating instances of type representations for that typeclass. To
illustrate this approach, the generic increment function will be
defined. This function increases by one the value of every integer
that occurs in a type. This is expressed by the following class:
\begin{hscode}\SaveRestoreHook
\column{B}{@{}>{\hspre}l<{\hspost}@{}}%
\column{3}{@{}>{\hspre}l<{\hspost}@{}}%
\column{E}{@{}>{\hspre}l<{\hspost}@{}}%
\>[B]{}\mathbf{class}\;\Conid{GInc}\;\Varid{r}\;\mathbf{where}{}\<[E]%
\\
\>[B]{}\hsindent{3}{}\<[3]%
\>[3]{}\Varid{gInc}\mathbin{::}\Varid{r}\to \Varid{r}{}\<[E]%
\ColumnHook
\end{hscode}\resethooks
and is implemented as follows:
\begin{hscode}\SaveRestoreHook
\column{B}{@{}>{\hspre}l<{\hspost}@{}}%
\column{3}{@{}>{\hspre}l<{\hspost}@{}}%
\column{E}{@{}>{\hspre}l<{\hspost}@{}}%
\>[B]{}\mathbf{instance}\;\Conid{GInc}\;(\Conid{K}\;\Conid{Int})\;\mathbf{where}{}\<[E]%
\\
\>[B]{}\hsindent{3}{}\<[3]%
\>[3]{}\Varid{gInc}\;(\Conid{K}\;\Varid{i})\mathrel{=}\Conid{K}\;(\Varid{i}\mathbin{+}\mathrm{1}){}\<[E]%
\\[\blanklineskip]%
\>[B]{}\mathbf{instance}\;\Conid{GInc}\;\Conid{Unit}\;\mathbf{where}{}\<[E]%
\\
\>[B]{}\hsindent{3}{}\<[3]%
\>[3]{}\Varid{gInc}\;\Conid{Unit}\mathrel{=}\Conid{Unit}{}\<[E]%
\\[\blanklineskip]%
\>[B]{}\mathbf{instance}\;\Conid{GInc}\;\Conid{Id}\;\mathbf{where}{}\<[E]%
\\
\>[B]{}\hsindent{3}{}\<[3]%
\>[3]{}\Varid{gInc}\;(\Conid{Id}\;\Varid{r})\mathrel{=}\Conid{Id}\mathbin{\$}\Varid{from}\mathbin{\$}\Varid{gInc}\mathbin{\$}\Varid{to}\;\Varid{r}{}\<[E]%
\\[\blanklineskip]%
\>[B]{}\mathbf{instance}\;(\Conid{GInc}\;\Varid{f},\Conid{GInc}\;\Varid{g})\Rightarrow \Conid{GInc}\;(\Varid{f}\;\oplus\;\Varid{g})\;\mathbf{where}{}\<[E]%
\\
\>[B]{}\hsindent{3}{}\<[3]%
\>[3]{}\Varid{gInc}\;(\Conid{Inl}\;\Varid{f})\mathrel{=}\Conid{Inl}\mathbin{\$}\Varid{gInc}\;\Varid{f}{}\<[E]%
\\
\>[B]{}\hsindent{3}{}\<[3]%
\>[3]{}\Varid{gInc}\;(\Conid{Inr}\;\Varid{g})\mathrel{=}\Conid{Inr}\mathbin{\$}\Varid{gInc}\;\Varid{g}{}\<[E]%
\\[\blanklineskip]%
\>[B]{}\mathbf{instance}\;(\Conid{GInc}\;\Varid{f},\Conid{GInc}\;\Varid{g})\Rightarrow \Conid{GInc}\;(\Varid{f}\;\otimes\;\Varid{g})\;\mathbf{where}{}\<[E]%
\\
\>[B]{}\hsindent{3}{}\<[3]%
\>[3]{}\Varid{gInc}\;(\Varid{f}\;\otimes\;\Varid{g})\mathrel{=}\Varid{gInc}\;\Varid{f}\;\otimes\;(\Varid{gInc}\;\Varid{g}){}\<[E]%
\\[\blanklineskip]%
\>[B]{}\mathbf{instance}\;\Conid{GInc}\;(\Conid{K}\;\Varid{a})\;\mathbf{where}{}\<[E]%
\\
\>[B]{}\hsindent{3}{}\<[3]%
\>[3]{}\Varid{gInc}\;\Varid{x}\mathrel{=}\Varid{x}{}\<[E]%
\ColumnHook
\end{hscode}\resethooks
This definition requires the Overlapping Instances extension (among
others) since there is no way to provide a specific case for \ensuremath{\Conid{Int}} and
a case for everything but \ensuremath{\Conid{Int}}. It can be observed that the \ensuremath{\Conid{GInc}}
function works for any representation given in terms of Regular's
types. To finalize the definition, a wrapper must be written:
\begin{hscode}\SaveRestoreHook
\column{B}{@{}>{\hspre}l<{\hspost}@{}}%
\column{E}{@{}>{\hspre}l<{\hspost}@{}}%
\>[B]{}\Varid{inc}\mathrel{=}\Varid{from}\;\circ\;\Varid{gInc}\;\circ\;\Varid{to}{}\<[E]%
\ColumnHook
\end{hscode}\resethooks
This wrapper simply converts the value into the representation and
then converts the result back to a value.

\end{document}
