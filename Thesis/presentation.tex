\documentclass{beamer}
\usepackage{graphicx}
\usepackage{multicol}
\usepackage{color}

\newcommand{\pos}{

    \includegraphics[scale=0.1]{plus.jpg}
}

\newcommand{\nig}{

  \includegraphics[scale=0.1]{minus.jpg}
}

\begin{document}

\begin{frame}

  \begin{tabular}{c}
    {\color{blue}\Huge Generic Programming in F\#} \\
    \\
    Ernesto Rodriguez \\
    Supervisor: Prof. Dr. Wouter Swierstra \\
    \\
    Software Technology Group\\
    Utrecht University
    
  \end{tabular}
  
\end{frame}

\begin{frame}
  \frametitle{Generic Programming and F\#}

  \begin{itemize}
  \item Some functions cannot be implemented generically due to the
    lack of type constructor polymorphism
  \item Many approaches exists in Haskell: Regular, RepLib, Multi-Rec
    and more
  \item Require advanced type system features such as
    kind-polymorphism
  \item Little research about generic programming has been done in F\#
  \item The only alternative to implement polytipic functions
    generically in F\# is reflection
  \end{itemize}
    
\end{frame}

\begin{frame}

  \frametitle{Generic Programming}

  \begin{itemize}
  \item Many functions, such as equality, have to be written over and
    over again (Jeuring and Jansson)
  \item This leads to code duplication and scalability problems
  \item Polymorphism can help mitigate the problem
  \item In many cases, polymorphism isn't enough: pattern matching
    cannot be done generically!
  \item Generic programming was developed to solve the problem
  \item Combinator based libraries and datatype generic programming
  \end{itemize}

  \centering \includegraphics[scale=0.4]{pattern} \\
    
\end{frame}

\begin{frame}

  \frametitle{Datatype Generic Programming}
  \begin{itemize}
  \item Will now be called generic programming
  \item Represent types and values using a representation type $U$
  \item Provide functions $from : Set \rightarrow U$ and $to : U
    \rightarrow Set$ to convert between values and representations
  \item Define functions on values of the representation type instead
    of ordinary values
  \item Values can be constructed generically
  \item Many approaches exist for Haskell: Regular, RepLib and
    Multi-Rec (among others)
  \item Each differs in the family of types that it can represent
  \item In general, the bigger the family the more complex the library
  \item This thesis is based on Regular due to its simplicity
  \end{itemize}
  
\end{frame}

\begin{frame}
  \frametitle{Regular}

  \begin{itemize}
    \item Types are made an instance of the Regular typeclass by
      specifying the representation of the type and the conversion
      functions
    \item Can be done automatically using Template Haskell
    \item Shallow representations
  \end{itemize}
  
  \begin{tabular}{ll}
    \includegraphics[scale=0.4]{rep1} & \includegraphics[scale=0.4]{reg} \\
    \includegraphics[scale=0.4]{rep3} & \\
  \end{tabular}
\end{frame}


\begin{frame}

  \frametitle{Regular: Example}


  \begin{multicols}{2}
    \begin{itemize}
    \item Generic functions are defined as typeclasses
    \item The individual constituents of the representation type are
      made an instance of the typeclass
    \item Compiler typechecks functions by instantiating type
      variables
    %\item Variables have higher kinds
    \item This function requires the Overlapping Instances extension
    \end{itemize}
    \pagebreak
    \includegraphics[scale=0.4]{gsum}
  \end{multicols}
  
\end{frame}

\begin{frame}

  \frametitle{The F\# programming languages}

  \begin{itemize}
  \item Functional language of the ML-family (OCaml implementation for .NET).
  \item Syntax directed type inference based on Hindley-Miller.
  \item Inter-operates with other .NET languages $\rightarrow$ no type erasure.
  \item Parametric polymorphism with first order type variables.
  \item Ad-hoc polymorphism through interfaces and member constrains.
  \item Inheritance support for class types.
  \item Designed to be practical language that is easy to adopt by
    .NET programmers.
  \end{itemize}
\end{frame}

\begin{frame}

  \frametitle{The F\# programming language}
  
  \begin{tabular}{c}
    ADTs and Pattern Matching \\
    \includegraphics[scale=0.4]{adts} \\
    Classes and Inheritance \\
    \includegraphics[scale=0.4]{classes} \\
  \end{tabular}
  
\end{frame}

\begin{frame}

  \frametitle{Reflection}

  \begin{itemize}
  \item F\# runs on top of .NET and uses .NET's type system
  \item All types are available at run-time (no type erasure)
  \item Type information can be queried at run-time
  \item It is possible to inspect many constructs at runtime: methods, functions, types
  \item It is possible to perform almost any .NET operation
  \item Not type safe
  \item Requires (a lot) boilerplate code :(
  \item Requires knowledge about .NET to be used
  \item Not cross-platform
    
  \end{itemize}

  \centering\includegraphics[scale=0.4]{boilerplate}
  
\end{frame}

\begin{frame}

  \frametitle{Research}

  \begin{itemize}
  \item Investigate if datatype generic programming can be used in F\#
  \item Implement a library based on Regular
  \item Determine if F\#'s type system is powerful enough for generic programming
  \item Explore the advantages and disadvantages of this implementation
  \item Compare results with Regular and Reflection
    
  \end{itemize}
    
\end{frame}

\begin{frame}

  \frametitle{Representations in F\#}
  \begin{itemize}
  \item Typeclass constraints are converted into sub-type constraints
%  \item Sub-typeing is used to hide higher kinded variables
  \item The role of these representations mirrors that of Regular's
    types
  \end{itemize}
  
  \begin{tabular}{cc}
    \includegraphics[scale=0.4]{meta1} & \includegraphics[scale=0.4]{meta2} \\
  \end{tabular}
  
\end{frame}

\begin{frame}

  \frametitle{Automatic Conversion}

  \begin{itemize}

  \item Use the type information available at runtime to convert values into representations
  \item Every .NET value has the member function $GetType : unit \rightarrow Type$
  \item .NET defines the toplevel function $typeof\langle `t \rangle : Type$
  \item Every .NET language extends the $Type$ class in order to
    include metadata.
    
  \end{itemize}
\end{frame}

\begin{frame}
  \frametitle{Automatic Conversion}
  \centering\includegraphics[scale=0.4]{reflection}
\end{frame}

\begin{frame}

  \frametitle{Generic Functions as Classes}

  \begin{itemize}
  \item Generic functions are defined by overriding the
    \emph{FoldMeta} class
  \item The \emph{FoldMeta} class contains one overload for every
    sub-type of \emph{Meta}
  \item Contains three type arguments: the generic type (\emph{'ty}),
    the input type (\emph{'in}) and the output type (\emph{'out})
  \item Contains a member to dispatch other methods (more about it later)
  \end{itemize}
  
  \centering\includegraphics[scale=0.4]{foldmeta}
  
\end{frame}

\begin{frame}
  \frametitle{Example: GMap}
  \includegraphics[scale=0.4]{gmap}
\end{frame}

\begin{frame}
  \frametitle{The FoldMeta class: problem} Consider invoking
  \emph{gsum} on a value of type: $(K\ Int\ \otimes\ I)\ \oplus\ U$
  \begin{itemize}
  \item Is the definition correct? How can the compiler check?
  \item What \emph{gsum} overload must be selected? Input could be of
    type $U \oplus U$ or $GSum\ a \Rightarrow a \oplus U$.
  \item Can F\# do it this way?
  \end{itemize}
  \centering\includegraphics[scale=0.4]{pgsum}
\end{frame}

\begin{frame}
  \frametitle{The FoldMeta class: solution}
  \begin{itemize}
  \item Require that all cases are covered
  \item All recursive calls are performed on values of type
    \emph{Meta}
  \item Select an overload using reflection: specific type $>$
    polymorphic type
  \item All overloads must return the same type
  \item Generic functions must be total (for the universe)
  \item Instead of abstracting using type variables we use subtypeing
  \end{itemize}
  \centering\includegraphics[scale=0.4]{foldmeta}
    
\end{frame}

\begin{frame}
  \frametitle{What about Ad-Hoc polymorphism?}
  \begin{itemize}
  \item Extension members are not checked when solving member
    constrains
  \item The signature of type extensions must be the same as the
    original signature
  \item Method definitions might overlap
  \end{itemize}
  If the restrictions above didn't exists, extension members would
  be very similar to typeclasses.

  \centering\includegraphics[scale=0.4]{adhoc}
  
\end{frame}

\begin{frame}

  \frametitle{Type Providers}

  An F\# tool that allows types to be generated at compile time by
  executing arbitrary .NET code. However:
  
  \begin{itemize}
  \item Can only generate isomorphic types
  \item Only accept strings, integers and booleans as static
    arguments.
    
  \end{itemize}

  \centering\includegraphics[scale=0.4]{provider}
  
\end{frame}

\begin{frame}
  \frametitle{Extensibility}
  \begin{itemize}
  \item It is trivial to modify the functionality in a modular way
  \item Generic functions and their variants can easily co-exist in
    the same namespace
  \item Method dispatch is easy to customize
  \item Overriding is natural for OOP (FP prefers abstraction)
  \end{itemize}
  
  \centering\includegraphics[scale=0.4]{extensible}
\end{frame}

\begin{frame}
  \frametitle{Multi-Argument Recursion}
  \begin{itemize}
  \item \emph{FoldMeta} variant that takes more representation
    arguments
  \item Selection works in a similar way as before
  \end{itemize}
  
  \includegraphics[scale=0.4]{mrec}
  
\end{frame}

\begin{frame}

  \frametitle{Regular vs FoldMeta}

  \begin{tabular}{l p{4cm} l p{4cm}}
    \hline
    \multicolumn{2}{c}{Regular} & \multicolumn{2}{c}{FoldMeta} \\
    \hline

    \pos & Enforces representations to always be valid & \nig & Ill
    formed representations result in runtime errors \\

    \pos & Partial generic functions & \nig & Generic functions must
    be total \\

    \pos & Different cases of a generic function can return different
    types & \nig & All cases of a generic function must return the same type \\

    \nig & Hard to modify generic functions & \pos & Easy to
    modify generic functions \\

    \nig & Requires the Overlapping Instances extension to select
    overlapping overloads & \pos & Uses inheritance to customize
    overload selection \\
    
  \end{tabular}
  
\end{frame}

\begin{frame}
  \frametitle{Conclusions}

  \begin{itemize}

    \item Datatype generic programming heavily relies on higher kinds
      and their absence has many shortcomings (esp. when generating
      values generically)

    \item F\# could adopt ideas from generic programming since the
      only alternative available at the moment is reflection

    \item Datatype generic programming can benefit from the
      OO-paradigm

    \item Type providers and ad-hoc polymorphism do not suffice for
      Datatype Generic Programming

    \item Reflection is able to supplant the typelevel computations
      performed by the Haskell compilers

    \item This library is a lightweight alternative to Reflection
      
  \end{itemize}
  
\end{frame}

\begin{frame}
  \frametitle{Future Work}
  \begin{itemize}

    \item Reflection can be used to perform static analysis on
      compiled assemblies for extra type safety
    \item Function calls can be cached to reduce the runtime penalty
      of reflection
    \item Consider other approaches for generic programming in F\#
      since the advantages of Datatype Generic Programming are absent
      without kind-polymorphism and typeclasses
  \end{itemize}
  
\end{frame}

\begin{frame}

  \centering{\Huge Thank you for listening}
  \vspace{2cm}
  \\Special thanks to Dr. Wouter Swierstra for his guidance and advice
  while performing this research and the reading club of the Software
  Technology Group for reviewing a draft of this work.

\end{frame}

\end{document}
