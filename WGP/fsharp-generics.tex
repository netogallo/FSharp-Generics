%-----------------------------------------------------------------------------
%
%               Template for sigplanconf LaTeX Class
%
% Name:         sigplanconf-template.tex
%
% Purpose:      A template for sigplanconf.cls, which is a LaTeX 2e class
%               file for SIGPLAN conference proceedings.
%
% Guide:        Refer to "Author's Guide to the ACM SIGPLAN Class,"
%               sigplanconf-guide.pdf
%
% Author:       Paul C. Anagnostopoulos
%               Windfall Software
%               978 371-2316
%               paul@windfall.com
%
% Created:      15 February 2005
%
%-----------------------------------------------------------------------------


\documentclass{sigplanconf}

% The following \documentclass options may be useful:

% preprint      Remove this option only once the paper is in final form.
% 10pt          To set in 10-point type instead of 9-point.
% 11pt          To set in 11-point type instead of 9-point.
% authoryear    To obtain author/year citation style instead of numeric.

\usepackage{amsmath}


\begin{document}

\special{papersize=8.5in,11in}
\setlength{\pdfpageheight}{\paperheight}
\setlength{\pdfpagewidth}{\paperwidth}

\conferenceinfo{CONF 'yy}{Month d--d, 20yy, City, ST, Country} 
\copyrightyear{20yy} 
\copyrightdata{978-1-nnnn-nnnn-n/yy/mm} 
\doi{nnnnnnn.nnnnnnn}

% Uncomment one of the following two, if you are not going for the 
% traditional copyright transfer agreement.

%\exclusivelicense                % ACM gets exclusive license to publish, 
                                  % you retain copyright

%\permissiontopublish             % ACM gets nonexclusive license to publish
                                  % (paid open-access papers, 
                                  % short abstracts)

\titlebanner{banner above paper title}        % These are ignored unless
\preprintfooter{short description of paper}   % 'preprint' option specified.

\title{Generic Programming in F\#}
\subtitle{Datatype generic programming for .Net}

\authorinfo{Ernesto Rodriguez}
           {Utrecht University}
           {e.rodriguez@students.uu.nl}
\authorinfo{Wouter Swierstra}
           {Utrecht University}
           {W.S.Swierstra@uu.nl}

\maketitle

\begin{abstract}
The introduction of Datatype Generic programming (DGP) *revolutionized* functional programming by allowing numerous algorithms to be defined by induction over the structure of types while still providing type safety. Due to the advanced type system requirements for DGP, only a handful of functional languages can define generic functions making it inaccessible to most programmers. Ordinary languages provide reflection and duck typing as a mechanism to specify generic algorithms. These mechanisms are usually error prone and verbose. By combining ideas from DGP and implementing them through reflection, a type-safe interface to DGP has been built for the F\# language. These generic algorithms can be accessed by any language running in the .Net platform.
\end{abstract}

\category{CR-number}{subcategory}{third-level}

% general terms are not compulsory anymore, 
% you may leave them out
\terms
term1, term2

\keywords
keyword1, keyword2

\section{Introduction}

\section{The F\# Language}
The F\# programming language is a functional language of the ML family. It focuses on being a productive language by levering on functional programming and at the same time easy to adopt by programmers of other .Net hosted languages. As a result, the lenguage has a much simpler type system than Haskell or Scala. Most of the development effort in the language has focused on features to work with data (like type-providers) and to be compatible with the .Net type system. Unlike Scala, F\# performs no type errasure when compiled to the .Net platform.
\\\\
There are several mechanism to define new types in F\#: classes, records and algebraic data types. Classes correspond to the traditional object oriented paradigm and are allowed to inherit fields and functions from another type as long as the type is not sealed (which is a .Net attribute for types). Records and algebraic datatypes correspond to the functional approach of defining types. Records and ADTs are always sealed and can be pattern matched. All types in F\# can define member functions (methods) and can implement any number of interfaces. Types can also have generic type arguments but they are required to be of kind $*$ (star).
\section{The .Net platform}
The .Net platform is a common runtime environment to allow the execution of a family of languages. It implements a very rich type system which includes support for generics. Many type operations that happen in F\# (such as sub-typeing) are handled by the .Net platform. The sub-typeing relation will be denoted by $\tau_a :> \tau_b$ which means $\tau_a$ is a sub-type of $\tau_b$ and consequently a value of type $\tau_a$ can be automatically converted to a value of type $\tau_b$.
\\\\
Like most object oriented langagues, .Net sub-typeing mechanism that allows types to be automatically converted to types which are higher in the class hierarchy. A well known restriction of this mechanism is that sub-typeing rules cannot automatically be applied to generic type arguments. In other words $\tau_a :> \tau_b\ \not\Rightarrow\ T<\tau_a> \ :> \ T<\tau_b>$.
\\\\
The .Net platform internally uses an abstract class \verb+Type+ to represent any of the types that are available. This class allows operations such as casting or instantiating the generic type arguments of a type.

The F\# language does not support generics of higher kind. This means that generic types in F\# cannot be applied to other types. This feature is used by other DGP \cite{Regular,MultiRec,GenericDeriving,RepLib} to enforce type safety and allow the compiler to select  

\appendix
\section{Appendix Title}

This is the text of the appendix, if you need one.

\acks

Acknowledgments, if needed.

% We recommend abbrvnat bibliography style.

\bibliographystyle{abbrvnat}

% The bibliography should be embedded for final submission.

\begin{thebibliography}{}
\softraggedright

\bibitem[Smith et~al.(2009)Smith, Jones]{smith02}
P. Q. Smith, and X. Y. Jones. ...reference text...

\end{thebibliography}


\end{document}

%                       Revision History
%                       -------- -------
%  Date         Person  Ver.    Change
%  ----         ------  ----    ------

%  2013.06.29   TU      0.1--4  comments on permission/copyright notices

